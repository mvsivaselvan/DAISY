\documentclass{article}

\usepackage{mathtools, amssymb, amsxtra}

\newcommand{\kv}{k_\text{v}}
\newcommand{\kr}{k_\text{r}}
\newcommand{\cv}{c_\text{v}}
\newcommand{\crr}{c_\text{r}}

\title{Pendulum dynamics with horizontal and vertical excitation}
\author{M.~V.~Sivaselvan, \texttt{mvs@buffalo.edu}}
\date{Oct 4, 2024}

\begin{document}

\maketitle

The following describes the dynamics of a pendulum (a rigid body) with both 
horizontal and vertical accelerations applied at the support.
This is a model for problems of much larger dimension I am interested in.
We use the state vector
\begin{equation*}
    \begin{aligned}
        x_1 &= \Delta\; \text{: vertical displacement} \\
        x_2 &= \theta\; \text{: rotation} \\
        x_3 &= \dot{\Delta} \\
        x_4 &= \dot{\theta}
    \end{aligned}
\end{equation*}
to describe the dynamics. The input accelerations in the horizontal and vertical 
directions are $u_x$ and $u_z$. The differential equation of motion is
\begin{equation*}
    \begin{aligned}
        m\ddot{\Delta} -m h \sin(\theta)\ddot{\theta} 
            -m h \cos(\theta) \dot{\theta}^2 + \kv\Delta + m g (1+u_z(t)) + \cv\dot{\Delta} &= 0\\
        -m h \sin(\theta)\ddot{\Delta} + I\ddot{\theta}
            + \kr\theta - m g (1+u_z(t)) h \sin(\theta) 
            + m g u_x(t) h \cos(\theta) + \crr\dot{\theta} &= 0
    \end{aligned}
\end{equation*}
Abbreviate these two equations as
\begin{equation*}
    F\left(t,
           \begin{pmatrix}\Delta \\ \theta\end{pmatrix},
           \begin{pmatrix}\dot{\Delta} \\ \dot{\theta}\end{pmatrix},
           \begin{pmatrix}\ddot{\Delta} \\ \ddot{\theta}\end{pmatrix}
           \right) = 0
\end{equation*}
The residual for IDA is 
\begin{equation*}
    \text{res}(t,x,\dot{x}) = \begin{pmatrix} 
                                  \dot{x}(1:2) - x(3:4) \\
                                  F\left(t,x(1:2),x(3:4),\dot{x}(3:4)\right)
                              \end{pmatrix}
\end{equation*}
The Jabobian of the residual contains the matrices
\begin{equation*}
    M = \begin{bmatrix} 
            \frac{\partial F_1}{\partial \ddot{\Delta}} &
            \frac{\partial F_1}{\partial \ddot{\theta}} \\[0.5em]
            \frac{\partial F_2}{\partial \ddot{\Delta}} &
            \frac{\partial F_2}{\partial \ddot{\theta}}
        \end{bmatrix}; \quad
    C = \begin{bmatrix} 
            \frac{\partial F_1}{\partial \dot{\Delta}} &
            \frac{\partial F_1}{\partial \dot{\theta}} \\[0.5em]
            \frac{\partial F_2}{\partial \dot{\Delta}} &
            \frac{\partial F_2}{\partial \dot{\theta}}
        \end{bmatrix}; \quad
    K = \begin{bmatrix} 
            \frac{\partial F_1}{\partial \Delta} &
            \frac{\partial F_1}{\partial \theta} \\[0.5em]
            \frac{\partial F_2}{\partial \Delta} &
            \frac{\partial F_2}{\partial \theta}
        \end{bmatrix}
\end{equation*}
Solving a linear equation with the Jacobian reduces to solving a linear equation 
with the matrix
\begin{equation*}
    c_j^2 M + c_j C + K
\end{equation*}
which has a sparse structure corresponding to a finite element stiffness matrix,
hence my interest in using a custom linear solver with IDA.
\end{document}